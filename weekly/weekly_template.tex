\documentclass[letterpaper, 11pt]{article}
\usepackage[utf8]{inputenc}
\usepackage{mathptmx}
\usepackage{natbib}
\usepackage[top=1in, bottom=1.25in, left=1.25in, right=1.25in]{geometry}
\usepackage{enumitem}
\usepackage[usenames,dvipsnames]{xcolor}
\usepackage{hyperref}
\hypersetup{colorlinks=true,allcolors=MidnightBlue}
\usepackage{titlesec}
\usepackage{graphicx}
\usepackage{wrapfig}
\usepackage{enumitem}
\usepackage{array}
\usepackage{booktabs}
\usepackage{amsmath}
\usepackage{amssymb}
%%%%%%%%%%%%%%%
%% FEEL FREE TO ADD MORE PACKAGES %%% 
%%%%%%
%%% LaTeX 
%% https://www.overleaf.com/learn/latex/Learn_LaTeX_in_30_minutes
%% https://www.overleaf.com/learn/latex/Positioning_images_and_tables 

%%% TODO: change TOPIC to your title
\title{SU2024 CS416 Data Visualisation: Essay for Narrative Visualisation}
%%% TODO: change YOURNAME and yournetid to your name and NetId
\author{Ke Xu\\ kex7@illinois.edu}
% \date{December 10, 2024}
\begin{document}

\maketitle

\begin{abstract}
This document is the essay of the Narrative Visualisation project of CS416 Data Visualisation.
\end{abstract}

\section{Messaging}
\label{sec:Messaging}
% The Introduction should explain the topic and research question (if you have one) for your review. Try to motivate your topic/question: why would somebody care about this topic and question? 


\subsection{What is the message you are trying to communicate with the narrative visualization?}
\label{subsec:1}
\paragraph{} The visualization is utilizing the U.S. Music revenue by year and type data from 1973 to 2018, and is trying to provide first an overview on the generated revenue, then
drill down to show the details for each year for all types, the webpage is designed to be interactive on users' demand when they try to get more details from the chart.


\section{Narrative Structure}
\label{sec:NarrativeStructure}
\subsection{Which structure was your narrative visualization designed to follow (martini glass, interactive slide show or drop-down story)?}
\label{subsec:NarrativeStructure1}

\subsection{How does your narrative visualization follow that structure? (All of these structures can include the opportunity to "drill-down" and explore. The difference is where that opportunity happens in the structure.)}
\label{subsec:NarrativeStructure2}


\section{Visual Structure}
\label{sec:VisualStructure}
\subsection{What visual structure is used for each scene?}
\label{subsec:VisualStructure1}
\subsection{How does it ensure the viewer can understand the data and navigate the scene?}
\label{subsec:VisualStructure2}

\subsection{How does it help the viewer transition to other scenes, to understand how the data connects to the data in other scenes?}
\label{subsec:VisualStructure3}

\subsection{How does it highlight to urge the viewer to focus on the important parts of the data in each scene?}
\label{subsec:VisualStructure4}


\section{Scenes}
\label{sec:Scenes}
\subsection{What are the scenes of your narrative visualization?}
\label{subsec:Scenes1}
\subsection{How are the scenes ordered, and why?}
\label{subsec:Scenes2}


\section{Annotations}
\label{sec:Annotations}
\subsection{What template was followed for the annotations, and why that template?}
\label{subsec:Annotations1}
\subsection{How are the annotations used to support the messaging?}
\label{subsec:Annotations2}
\subsection{Do the annotations change within a single scene, and if so, how and why}
\label{subsec:Annotations3}

\section{Parameters}
\label{sec:Parameters}
\subsection{What are the parameters of the narrative visualization?}
\label{subsec:Parameters1}
\subsection{What are the states of the narrative visualization?}
\label{subsec:Parameters2}
\subsection{How are the parameters used to define the state and each scene?}
\label{subsec:Parameters3}


\section{Triggers}
\label{sec:Triggers}
\subsection{What are the triggers that connect user actions to changes of state in the narrative visualization?}
\label{subsec:Triggers1}
\subsection{What affordances are provided to the user to communicate to them what options are available to them in the narrative visualization?}
\label{subsec:Triggers2}



% \bibliographystyle{acl_natbib}
% \bibliography{main}
\end{document}
